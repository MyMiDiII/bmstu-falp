\chapter{Практические задания}

\section{Задание №1}

Написать функцию, которая принимает целое число и возвращает первое четное
число, не меньшее аргумента.

\vspace{4mm}
\hfill
\begin{minipage}{0.92\linewidth}
\begin{lstlisting}
(defun first-greater-even (num)
    (if (evenp num) num (+ num 1)))
\end{lstlisting}
\end{minipage}


\section{Задание №2}

Написать функцию, которая принимает число и возвращает число того же знака, но с
модулем на 1 больше модуля аргумента.

\vspace{4mm}
\begin{minipage}{0.92\linewidth}
\begin{lstlisting}
(defun plus-minus-one (num)
    (+ num (if (> num 0) 1 -1)))
\end{lstlisting}
\end{minipage}

\section{Задание №3}

Написать функцию, которая принимает два числа и возвращает список из этих чисел,
расположенных по возрастанию.

\vspace{4mm}
\begin{minipage}{0.92\linewidth}
\begin{lstlisting}
(defun make-growing-list (a b)
    (if (< a b) (list a b) (list b a)))
\end{lstlisting}
\end{minipage}

\section{Задание №4}

Написать функцию, которая принимает три числа и возвращает Т только тогда, когда
первое число расположено между вторым и третьим.

\vspace{4mm}
\begin{minipage}{0.92\linewidth}
\begin{lstlisting}
(defun is-between (a b c)
    (or (and (> a b) (< a c)) (and (> a c) (< a b))))
\end{lstlisting}
\end{minipage}

\section{Задание №5}

Каков результат вычисления следующих выражений?

\vspace{4mm}
\begin{minipage}{0.92\linewidth}
\begin{lstlisting}
(and 'fee 'fie 'foe)          ; foe
(or nil 'fie 'foe)            ; fie
(and (equal 'abc 'abc) 'yes)  ; yes 
(or 'fee 'fie 'foe)           ; fee
(and nil 'fie 'foe)           ; nil
(or (equal 'abc 'abc) 'yes)   ; T
\end{lstlisting}
\end{minipage}

\section{Задание №6}

Написать предикат, который принимает два числа-аргумента и возвращает Т, если
первое число не меньше второго.

\vspace{4mm}
\begin{minipage}{0.92\linewidth}
\begin{lstlisting}
(defun pred (a b)
    (>= a b))
\end{lstlisting}
\end{minipage}

\section{Задание №7}

Какой из следующих двух вариантов предиката ошибочен и почему?

\vspace{4mm}
\begin{minipage}{0.92\linewidth}
\begin{lstlisting}
(defun pred1 (x)
    (and (numberp x) (plusp x)))

(defun pred2 (x)
    (and (plusp x) (numberp x)))
\end{lstlisting}
\end{minipage}

Ошибочным является второй вариант, так как функция plusp принимает на вход один
аргумент типа number, из-за чего аргументы, не являющиеся числами, будут
вызывать ошибку. При этом в первом варианте, если при проверке, является ли
аргумент числом, получится значение Nil, то and вернет его в качестве
результата, не продолжая дальнейшие вычисления, а при передачи числа будет
проведена проверка на положительность.

\section{Задание №8}

Решить задачу 4, используя для ее решения конструкции IF, COND, AND/OR.

\vspace{4mm}
\begin{minipage}{0.92\linewidth}
\begin{lstlisting}
;; if
(defun is-between-if (a b c)
    (if (> a b) (< a c) (> a c)))

;; cond
(defun is-between-cond (a b c)
    (cond
        ((> a b) (< a c))
        ((< a b) (> a c))))

;; and/or
(defun is-between (a b c)
    (or (and (> a b) (< a c)) (and (> a c) (< a b))))
\end{lstlisting}
\end{minipage}

\section{Задание №9}

Переписать функцию how-alike, приведенную в лекции и использующую COND,
используя только конструкции IF, AND/OR.

\vspace{4mm}
\begin{minipage}{0.92\linewidth}
\begin{lstlisting}
;; cond
(defun how-alike (x y)
    (cond ((or (= x y) (equal x y)) 'the_same)
          ((and (oddp x) (oddp y)) 'both_odd)
          ((and (evenp x) (evenp y)) 'both_even)
          (t 'difference)))

;; if
(defun how-alike-if (x y)
    (if (or (= x y) (equal x y)) 'the_same
        (if (and (oddp x) (oddp y)) 'both_odd
            (if (and (evenp x) (evenp y)) 'both_even
                'difference))))

;; and/or
(defun how-alike-and-or (x y)
    (or (and (or (= x y) (equal x y)) 'the_same)
        (and (and (oddp x) (oddp y)) 'both_odd)
        (and (and (evenp x) (evenp y)) 'both_even)
        'difference))
\end{lstlisting}
\end{minipage}

\chapter{Теоретические вопросы}

\section{Базис языка Lisp}

Базис языка --- минимальный набор конструкций и структур данных, с помощью
которого можно решить любую задачу.

Базис Lisp образуют:
\begin{itemize}
    \item атомы;
    \item структуры;
    \item базовые функции (atom, eq, cons, car, cdr);
    \item базовые специальные функции и  функционалы (cond, quote, lambda, eval).
\end{itemize}

\section{Классификация функций}

Функции классифицируются на:

\begin{itemize}
    \item чистые функции;
    \item рекурсивные;
    \item специальные функции или формы;
    \item псевдофункции;
    \item функции с вариантами значений;
    \item функционалы.
\end{itemize}

Классификация базовых функций:
\begin{itemize}
    \item селекторы (car, cdr);
    \item конструкторы (cons, list);
    \item предикаты (null, consp, ...);
\end{itemize}


\section{Способы создания функций}

Лямбда определения:

\vspace{4mm}
\begin{minipage}{0.92\linewidth}
\begin{lstlisting}
(lambda <lambda-список> <форма>) ; lambda-выражение
;; <lambda-список> -- список аргументов
;; <форма> -- тело функции
\end{lstlisting}
\end{minipage}

Определение функций с именем:

\vspace{4mm}
\begin{minipage}{0.92\linewidth}
\begin{lstlisting}
(defun <имя> <lambda-выражение>)
\end{lstlisting}
\end{minipage}

\section{Работа функций cond, if, and/or}

\subsection{Функция cond}

Синтаксис:

\vspace{4mm}
\begin{minipage}{0.92\linewidth}
\begin{lstlisting}
(cond
    (test1 body1)
    (test2 body2)
    ...
    (testN bodyN)
    [(T else-body)])
\end{lstlisting}
\end{minipage}

По порядку вычисляются и проверяются на равенство с Nil предикаты. Для
первого предиката, который не равен Nil, вычисляется находящееся с ним
в списке выражение и возвращается его значение. Если все предкаты вернут
Nil, то и cond вернет Nil. Ветка <<else>> организуется явным указанием в
качестве test --- T.

\subsection{Функция if}

Синтаксис:

\vspace{4mm}
\begin{minipage}{0.92\linewidth}
\begin{lstlisting}
(if test t-body f-body)
\end{lstlisting}
\end{minipage}

Если вычисленный предикат не Nil, то выполняется t-body, иначе ---
\mbox{f-body}.

\subsection{Функция and}

Синтаксис:

\vspace{4mm}
\begin{minipage}{0.92\linewidth}
\begin{lstlisting}
(and arg1 arg2 ... argN)
\end{lstlisting}
\end{minipage}

Функция возвращает Nil при встрече первого (при вычислении слева направо)
аргумента со значением Nil. Если
все не Nil, то возвращается результат вычисления последнего аргумента.

\subsection{Функция or}

Синтаксис:

\vspace{4mm}
\begin{minipage}{0.92\linewidth}
\begin{lstlisting}
(or arg1 arg2 ... argN)
\end{lstlisting}
\end{minipage}

Функция возвращает первый arg\_i, результат вычисления которого не
Nil. Если все Nil, то возвращается Nil.
