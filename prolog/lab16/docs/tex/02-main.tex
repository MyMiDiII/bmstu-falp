\chapter{Практические задания}

\section{Задание}

Используя хвостовую рекурсию, разработать программу, позволяющую найти
\begin{enumerate}[label=\arabic*)]
    \item n!,
    \item n-е число Фибоначчи.
\end{enumerate}

Убедиться в правильности результатов.  Для одного из вариантов ВОПРОСА и каждого
задания составить таблицу, отражающую конкретный порядок работы системы.

Т.к.  резольвента хранится в виде стека, то состояние резольвенты требуется
отображать в столбик: вершина – сверху! Новый шаг надо начинать с нового
состояния резольвенты!

\clearpage
\section{Текст программы}

\mylisting{lab16.pro}{12-55}

