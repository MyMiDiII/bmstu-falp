\chapter{Практические задания}

\vspace{-0.5cm}
\section{Задание. Часть 1}
\vspace{-0.5cm}

Составить программу, т.е. модель предметной области --- базу знаний, объединив в
ней информацию --- знания:

\begin{itemize}
    \item \textbf{«Телефонный справочник»}: Фамилия, №тел, Адрес --- структура
        (Город, Улица, №дома, №кв);
    \item \textbf{«Автомобили»}: Фамилия\_владельца, Марка, Цвет, Стоимость,
        и~др.;
    \item \textbf{«Вкладчики банков»}: Фамилия, Банк, счет, сумма и~др.
\end{itemize}

Владелец может иметь несколько телефонов, автомобилей, вкладов \mbox{(Факты)}.

Используя правила, обеспечить возможность поиска:
\begin{enumerate}[label=\arabic*)]
    \item \begin{enumerate}[label=\alph*)]
        \item по №телефона найти: Фамилию, Марку автомобиля, Стоимость
            автомобиля (может быть несколько),
        \item используя сформированное в пункте а) правило, по № телефона найти:
            только Марку автомобиля (может быть несколько);
    \end{enumerate}
    \item используя простой, не составной вопрос: по Фамилии (уникальна в
        городе, но в разных городах есть однофамильцы) и Городу проживания
        найти: Улицу проживания, Банки, в которых есть вклады и №телефона.
    \item для одного из вариантов ответов, и для a) и для b), описать словесно
        порядок поиска ответа на вопрос, указав, как выбираются знания, и, при
        этом, для каждого этапа унификации, выписать подстановку --- наибольший
        общий унификатор, и соответствующие примеры термов.
\end{enumerate}

\vspace{-0.5cm}
\section{Задание. Часть 2}
\vspace{-0.5cm}

Используя конъюнктивное правило и простой вопрос, обеспечить возможность поиска
по Марке и Цвету автомобиля найти Фамилию, Город, Телефон и Банки, в которых
владелец автомобиля имеет вклады. Владельцев может быть несколько (не более
3-х), один и ни одного.

\begin{enumerate}[label=\arabic*)]
    \item Для каждого из трех вариантов словесно подробно описать порядок
        формирования ответа (в виде таблицы). При этом, указать – отметить
        моменты очередного запуска алгоритма унификации и полный результат его
        работы.  Обосновать следующий шаг работы системы. Выписать унификаторы –
        подстановки. Указать моменты, причины и результат отката, если он есть.
    \item Для случая нескольких владельцев (2-х) приведите примеры (таблицы)
        работы системы при разных порядках следования в БЗ процедур, и знаний в
        них: «Телефонный справочник», «Автомобили», «Вкладчики банков» или
        «Автомобили», «Вкладчики банков», «Телефонный справочник». Сделайте
        вывод, одинаковы ли множество работ и объем работ в разных случаях?
    \item Оформите 2 таблицы, демонстрирующие порядок работы алгоритма
        унификации вопроса и подходящего заголовка правила (для двух случаев из
        пункта 2) и укажите результаты его работы: ответ и побочный эффект.
\end{enumerate}

\vspace{-0.5cm}
\section{Текст программы}
\vspace{-0.5cm}

\mylisting{lab12.pro}{13-94}

