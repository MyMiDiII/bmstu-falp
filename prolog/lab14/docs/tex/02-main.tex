\chapter{Практические задания}

\section{Задание}

Создать базу знаний: «ПРЕДКИ», позволяющую наиболее эффективным способом (за
меньшее количество шагов, что обеспечивается меньшим количеством предложений БЗ
--- правил), и используя разные варианты (примеры) одного вопроса, определить
(указать: какой вопрос для какого варианта):
\begin{enumerate}
    \item по имени субъекта определить всех его бабушек (предки 2-го колена),
    \item по имени субъекта определить всех его дедушек (предки 2-го колена),
    \item по имени субъекта определить всех его бабушек и дедушек (предки 2-го
        колена),
    \item по имени субъекта определить его бабушку по материнской линии (предки
        2-го колена),
    \item по имени субъекта определить его бабушку и дедушку по материнской
        линии (предки 2-го колена).
\end{enumerate}

Минимизировать количество правил и количество вариантов вопросов. Использовать
конъюнктивные правила и простой вопрос.

Для одного из вариантов ВОПРОСА и конкретной БЗ составить таблицу,
отражающую конкретный порядок работы системы, с объяснениями:

\begin{itemize}
    \item очередная проблема на каждом шаге и метод ее решения;
    \item каково новое текущее состояние резольвенты, как получено;
    \item какие дальнейшие действия? (Запускается ли алгоритм унификации? Каких
        термов?  Почему этих?) ;
    \item вывод по результатам очередного шага и дальнейшие действия.
\end{itemize}

Т.к. резольвента хранится в виде стека, то состояние резольвенты требуется
отображать в столбик: вершина --- сверху! Новый шаг надо начинать с нового
состояния резольвенты!

\clearpage
\section{Текст программы}

\mylisting{lab14.pro}{12-46}

