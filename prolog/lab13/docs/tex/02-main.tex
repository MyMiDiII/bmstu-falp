\chapter{Практические задания}

\section{Задание}

Создать базу знаний «Собственники», дополнив (и минимально изменив) базу
знаний, хранящую знания:
\begin{itemize}
    \item \textbf{«Телефонный справочник»}: Фамилия, №тел, Адрес --- структура
        (Город, Улица, №дома, №кв);
    \item \textbf{«Автомобили»}: Фамилия\_владельца, Марка, Цвет, Стоимость,
        и~др.;
    \item \textbf{«Вкладчики банков»}: Фамилия, Банк, счет, сумма и~др.
\end{itemize}
знаниями о дополнительной собственности владельца. Преобразовать знания об
автомобиле к форме знаний о собственности.

Вид собственности (кроме автомобиля):

\begin{itemize}
    \item Строение, стоимость и другие его характеристики;
    \item Участок, стоимость и другие его характеристики;
    \item Водный\_транспорт, стоимость и другие его характеристики.
\end{itemize}

Описать и использовать вариантный домен: Собственность. Владелец может иметь, но
только один объект каждого вида собственности (это касается и автомобиля), или
не иметь некоторых видов собственности.

Используя конъюнктивное правило и разные формы задания одного вопроса (пояснять
для какого №задания --- какой вопрос), обеспечить возможность поиска:

\begin{enumerate}[label=\arabic*)]
    \item Названий всех объектов собственности заданного субъекта;
    \item Названий и стоимости всех объектов собственности заданного субъекта.
    \item Разработать правило, позволяющее найти суммарную стоимость всех
        объектов собственности заданного субъекта.
\end{enumerate}

Для 2-го пункта и одной фамилии составить таблицу, отражающую конкретный порядок
работы системы, с объяснениями порядка работы и особенностей использования
доменов (указать конкретные Т1 и Т2 и полную подстановку на каждом шаге).

\clearpage
\section{Текст программы}

\mylisting{lab13.pro}{13-55}

