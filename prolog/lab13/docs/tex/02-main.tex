\chapter{Практические задания}

\vspace{-0.5cm}
\section{Задание}
\vspace{-0.5cm}

Составить программу, т.е. модель предметной области --- базу знаний, объединив в
ней информацию --- знания:

\begin{itemize}
    \item \textbf{«Телефонный справочник»}: Фамилия, №тел, Адрес --- структура
        (Город, Улица, №дома, №кв);
    \item \textbf{«Автомобили»}: Фамилия\_владельца, Марка, Цвет, Стоимость,
        и~др.;
    \item \textbf{«Вкладчики банков»}: Фамилия, Банк, счет, сумма и~др.
\end{itemize}

Владелец может иметь несколько телефонов, автомобилей, вкладов \mbox{(Факты)}.

Используя правила, обеспечить возможность поиска:
\begin{enumerate}[label=\arabic*)]
    \item \begin{enumerate}[label=\alph*)]
        \item по №телефона найти: Фамилию, Марку автомобиля, Стоимость
            автомобиля (может быть несколько),
        \item используя сформированное в пункте а) правило, по № телефона найти:
            только Марку автомобиля (может быть несколько);
    \end{enumerate}
    \item используя простой, не составной вопрос: по Фамилии (уникальна в
        городе, но в разных городах есть однофамильцы) и Городу проживания
        найти: Улицу проживания, Банки, в которых есть вклады и №телефона.
    \item для одного из вариантов ответов, и для a) и для b), описать словесно
        порядок поиска ответа на вопрос, указав, как выбираются знания, и, при
        этом, для каждого этапа унификации, выписать подстановку --- наибольший
        общий унификатор, и соответствующие примеры термов.
\end{enumerate}
\vspace{-0.5cm}
\section{Текст программы}
\vspace{-0.5cm}

\mylisting{lab13.pro}{13-55}

