\chapter{Практические задания}

\section{Задание №1}

Составить диаграмму вычисления следующих выражений:

\vspace{4mm}
\hfill
\begin{minipage}{0.81\linewidth}
\begin{lstlisting}
(equal 3 (abs - 3))
(equal (+ 1 2) 3)
(equal (* 4 7) 21)
(equal 3 (* 2 3) (+ 7 2))
(equal (- 7 3) (* 3 2))
(equal (abs (- 2 4)) 3)
\end{lstlisting}
\end{minipage}

Решение приложено к отчету на отдельном листе.

\section{Задание №2}

Написать функцию, вычисляющую гипотенузу прямоугольного треугольника по
заданным катетам и составить диаграмму её вычисления.

\vspace{4mm}
\begin{minipage}{0.92\linewidth}
\begin{lstlisting}
(defun hyp (x y) (sqrt (+ (* x x) (* y y))))
\end{lstlisting}
\end{minipage}

Диаграмма приложена к отчету на отдельном листе.

\section{Задание №3}

Написать функцию, вычисляющую объем параллелепипеда по 3-м его
сторонам, и составить диаграмму ее вычисления.

\vspace{4mm}
\begin{minipage}{0.92\linewidth}
\begin{lstlisting}
(defun volume (x y z) (* x y z))
\end{lstlisting}
\end{minipage}

Диаграмма приложена к отчету на отдельном листе.


\section{Задание №4}

Каковы результаты вычисления следующих выражений?

\vspace{4mm}
\begin{minipage}{0.92\linewidth}
\begin{lstlisting}
(list 'a c)                     
;; Результат
;; Ошибка: несвязанная переменная c
;; Устранение: добавление апострофа перед c

(cons 'a (b c))
;; Результат
;; Ошибка: неопределенная функция b, несвязанная переменная c
;; Устранение: добавление апострофа перед (b c)

(cons 'a '(b c))
;; Результат
;; (a b c)

(caddy (1 2 3 4 5))
;; Результат
;; Ошибка: неопределенная функция caddy,
;;         недопустимый вызов функции (имя -- цифра)
;; Устранение: caddy -> caddr,
;;             добавление апострофа перед (1 2 3 4 5)

(cons 'a 'b 'c)
;; Результат
;; Ошибка: недопустимое количество параметров
;; Устранение: переча 2-ух параметров вместо 3-ех

(list 'a (b c))
;; Результат
;; Ошибка: неопределенная функция b, несвязанная переменная c
;; Устранение: добавление апострофа перед (b c)

(list a '(b c))
;; Результат
;; Ошибка: несвязанная переменная a
;; Устранение: добавление апострофа перед a

(list (+ 1 '(length '(1 2 3))))
;; Результат
;; Ошибка: (length '(1 2 3)) не является числом (это список)
;; Устранение: удаление апострофа перед (length '(1 2 3))
\end{lstlisting}
\end{minipage}

\section{Задание №5}

Написать функцию longer\_then от двух списков-аргументов, которые возвращает T,
если первый аргумент имеет большую длину.

\vspace{4mm}
\begin{minipage}{0.92\linewidth}
\begin{lstlisting}
(defun longer_then (list1 list2)
    (> (length list1) (length list2)))
\end{lstlisting}
\end{minipage}


\section{Задание №6}

Каковы результаты вычисления следующих выражений?

\vspace{4mm}
\begin{minipage}{0.92\linewidth}
\begin{lstlisting}
;; Выражение                                Результат
(cons 3 (list 5 6))                         ; (3 5 6)
(list 3 'from 9 'lives (- 9 3))             ; (3 from 9 lives 6)
(+ (length for 2 too) (car '(21 22 23)))    ; ошибка: несвязанная
                                            ; переменная for
(cdr '(cons is short for ans))              ; (is short for and)
(car (list one two))                        ; ошибка: несвязанная
                                            ; переменная one
(cons 3 '(list 5 6))                        ; (3 list 5 6)
(car (list 'one 'two))                      ; (one)
\end{lstlisting}
\end{minipage}

\section{Задание №7}

Дана функция:

\vspace{4mm}
\begin{minipage}{0.92\linewidth}
\begin{lstlisting}
(defun mystery (x) (list (second x) (first x)))
\end{lstlisting}
\end{minipage}

Какие результаты вычисления следующих выражений?

\vspace{4mm}
\begin{minipage}{0.92\linewidth}
\begin{lstlisting}
(mystery (one two))
;; Результат
;; Ошибка: несвязанная переменная two, неопределенная функция one
;; При исправлении:
;; (mystery '(one two)) -> (two one)

(mystery (last one two))
;; Результат
;; Ошибка: несвязанная переменная one
;; При исправлении:
;; (mystery (last '(one two))) -> (nil two)

(mystery free)
;; Результат
;; Ошибка: несвязанная переменная free
;; При исправлении:
;; (mystery '(free)) -> (nil free)
\end{lstlisting}
\end{minipage}

\vspace{4mm}
\begin{minipage}{0.92\linewidth}
\begin{lstlisting}
(mystery one 'two)
;; Результат
;; Ошибка: несвязанная переменная one
;; При исправлении:
;; (mystery 'one 'two) неверное число аргументов
;; (mystery '(one two)) -> (two one)
\end{lstlisting}
\end{minipage}

\section{Задание №8}

Написать функцию, которая переводит температуру в системе Фаренгейта в
температуру по Цельсию.
Как бы назывался роман Р.~Брэдбери <<451 градус по Фаренгейту>> в системе по Цельсию?

\vspace{4mm}
\begin{minipage}{0.92\linewidth}
\begin{lstlisting}
(defun f-to-c (temp) 
    (* 5/9 (- temp 32.0)))

(f-to-c 451) ; 232.77779
\end{lstlisting}
\end{minipage}

\section{Задание №9}

Какие результаты вычисления следующих выражений?

\vspace{4mm}
\begin{minipage}{0.92\linewidth}
\begin{lstlisting}
;; Выражение                     Результат
(list 'cons t Nil)               ; (cons t nil)
(eval (list 'cons t Nil))        ; (T)
(eval (eval (list 'cons t Nil))) ; неопределенная функция T 
(apply #cons ''(t Nil))          ; ошибка: неверный формат
                                 ; комплексного числа
;; при исправлении
(apply #'cons ''(t nil))         ; (quote t nil)
(list 'eval Nil)                 ; (eval nil)
(eval Nil)                       ; (nil)
(eval (list 'eval Nil))          ; (nil)
\end{lstlisting}
\end{minipage}

\section{Дополнительное задание №1}

Написать функцию, вычисляющую катет по заданной гипотенузе и другому
катету прямоугольного треугольника, и составить диаграмму ее выражения.

\vspace{4mm}
\begin{minipage}{0.92\linewidth}
\begin{lstlisting}
(defun cathetus (hyp cath)
    (sqrt (- (* hyp hyp) (* cath cath))))
\end{lstlisting}
\end{minipage}

Диаграмма приложена к отчету на отдельном листе.

\section{Дополнительное задание №2}

Написать функцию, вычисляющую площадь трапеции по ее основаниям и
высоте, и составить диаграмму ее вычисления.

\vspace{4mm}
\begin{minipage}{0.92\linewidth}
\begin{lstlisting}
(defun trapez_area (a b h)
    (* (/ (+ a b) 2.0) h))
\end{lstlisting}
\end{minipage}

Диаграмма приложена к отчету на отдельном листе.


\chapter{Теоретические вопросы}

\section{Базис языка Lisp}

Базис языка --- минимальный набор конструкций и структур данных, с помощью
которого можно решить любую задачу.

Базис Lisp образуют:
\begin{itemize}
    \item атомы;
    \item структуры;
    \item базовые функции (atom, eq, cons, car, cdr);
    \item базовые специальные функции и  функционалы (cond, quote, lambda, eval).
\end{itemize}

\section{Классификация функций}

Функции классифицируются на:

\begin{itemize}
    \item чистые функции;
    \item рекурсивные;
    \item специальные функции или формы;
    \item псевдофункции;
    \item функции с вариантами значений;
    \item функционалы.
\end{itemize}

Классификация базовых функций:
\begin{itemize}
    \item селекторы (car, cdr);
    \item конструкторы (cons, list);
    \item предикаты (null, consp, ...);
\end{itemize}


\section{Способы создания функций}

Лямбда определения:

\vspace{4mm}
\begin{minipage}{0.92\linewidth}
\begin{lstlisting}
(lambda <lambda-список> <форма>) ; lambda-выражение
;; <lambda-список> -- список аргументов
;; <форма> -- тело функции
\end{lstlisting}
\end{minipage}

Определение функций с именем:

\vspace{4mm}
\begin{minipage}{0.92\linewidth}
\begin{lstlisting}
(defun <имя> <lambda-выражение>)
\end{lstlisting}
\end{minipage}

\section{Функции car и cdr}

Функции car и cdr --- базовые функции доступа к данным, принимающие точечную
пару или список в качестве аргумента.

car --- функция, обеспечивающая доступ к первому элементу.

cdr --- функция, обеспечивающая доступ ко всем элементам, кроме первого.

\section{Назначение и отличие в работе cons и list}

Функции cons и list --- конструкторы.

Функция cons создает списковую ячейку и расставляет 2 указателя на
аргументы.

Функция list принимает нефиксированное число аргументов и возвращает список,
создает столько списковых ячеек, сколько элементов.
