\chapter{Практические задания}

\vspace{4mm}
\hfill
\begin{minipage}{0.92\linewidth}
\begin{lstlisting}
\end{lstlisting}
\end{minipage}

\section{Задание №1}

Написать функцию, которая по своему списку-аргументу lst определяет, является
ли он палиндромом (то есть равны ли lst и (reverse lst)).

\vspace{4mm}
\hfill
\begin{minipage}{0.92\linewidth}
\begin{lstlisting}
(defun is-palindrome (lst)
    (equal lst (reverse lst)))
\end{lstlisting}
\end{minipage}

\section{Задание №2}

Написать предикат set-equal, который возвращает t, если два его
множества-аргумента содержат одни и те же элементы, порядок которых не имеет
значения.

\vspace{4mm}
\begin{minipage}{0.92\linewidth}
\begin{lstlisting}
(defun set-equal1 (set1 set2)
    (and (subsetp set1 set2) (subsetp set2 set1)))
\end{lstlisting}
\end{minipage}

\section{Задание №3}

Напишите свои необходимые функции, которые обрабатывают таблицу из 4-х точечных
пар (страна . столица) и возвращают по стране --- столицу, а по столице ---
страну.

\vspace{4mm}
\begin{minipage}{0.92\linewidth}
\begin{lstlisting}
(defun get-capital (table country)
    (cond ((null table) Nil)
          ((equal (caar table) country) (cdar table))
          (T (get-capital (cdr table) country))))
\end{lstlisting}
\end{minipage}

\vspace{4mm}
\begin{minipage}{0.92\linewidth}
\begin{lstlisting}
(defun get-country (table capital)
    (cond ((null table) Nil)
          ((equal (cdar table) capital) (caar table))
          (T (get-country (cdr table) capital))))
\end{lstlisting}
\end{minipage}

\section{Задание №4}

Напишите функцию swap-first-last, которая переставляет в списке-аргументе
первый и последний элементы.

\vspace{4mm}
\begin{minipage}{0.92\linewidth}
\begin{lstlisting}
(defun swap-first-last (lst)
    (let ((fst-el (car lst))
          (lst-el (last lst)))
         (setf (car lst) (car lst-el))
         (setf (car lst-el) fst-el))
         lst)
\end{lstlisting}
\end{minipage}

\section{Задание №5}

Напишите функцию swap-two-element, которая переставляет в списке-аргументе
два указанных своими порядковыми номерами элемента в этой списке.

\vspace{4mm}
\begin{minipage}{0.92\linewidth}
\begin{lstlisting}
(defun swap-two-element (lst num1 num2)
    (let ((len (length lst)))
         (and (< num1 len) (< num2 len)
             (let ((fst-el (nth num1 lst))
                   (snd-el (nth num2 lst)))
                  (setf (nth num1 lst) snd-el)
                  (setf (nth num2 lst) fst-el))
                  lst)))
\end{lstlisting}
\end{minipage}

\section{Задание №6}

Напишите две функции swap-to-left и swap-to-right, которые производят одну
круговую перестановку в списке-аргументе влево и вправо, соответственно.

\vspace{4mm}
\begin{minipage}{0.92\linewidth}
\begin{lstlisting}
(defun swap-to-left (lst)
     (append (cdr lst) (cons (car lst) Nil)))
\end{lstlisting}
\end{minipage}

\vspace{4mm}
\begin{minipage}{0.92\linewidth}
\begin{lstlisting}
(defun swap-to-right (lst)
     (cons (car (last lst)) (reverse (cdr (reverse lst)))))
\end{lstlisting}
\end{minipage}

\section{Задание №7}

Напишите функцию, которая добавляет к множеству двухэлементных списков
новый двухэлементный список, если его там нет.

\vspace{4mm}
\begin{minipage}{0.92\linewidth}
\begin{lstlisting}
\end{lstlisting}
\end{minipage}

\section{Задание №8}

Напишите функцию, которая умножает на заданное число-аргумент первый
числовой элемент списка из заданного 3-хэлементного списка-аргумента
когда:

\begin{enumerate}
    \item все элементы списка --- числа,

\vspace{4mm}
\begin{minipage}{0.92\linewidth}
\begin{lstlisting}
\end{lstlisting}
\end{minipage}

    \item элементы списка --- любые объекты.

\vspace{4mm}
\begin{minipage}{0.92\linewidth}
\begin{lstlisting}
\end{lstlisting}
\end{minipage}

\end{enumerate}

\section{Задание №9}

Напишите функцию select-between, которая из списка-аргумента из 5 чисел
выбирает только те, которые расположены между двумя указанными
границами-аргументами и возвращет их в виде списка (упорядоченного по
возрастанию списка чисел).

\vspace{4mm}
\begin{minipage}{0.92\linewidth}
\begin{lstlisting}
\end{lstlisting}
\end{minipage}

\chapter{Теоретические вопросы}

\section{Структуроразрушающие и не разрушающие структуру списка
функции}

\section{Отличие в работе функций cons, list, append, nconc и в их результате}
