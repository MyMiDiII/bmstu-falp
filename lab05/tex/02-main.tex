\chapter{Практические задания}

\section{Задание №1}

Написать функцию, которая по своему списку-аргументу lst определяет, является
ли он палиндромом (то есть равны ли lst и (reverse lst)).

\vspace{4mm}
\hfill
\begin{minipage}{0.92\linewidth}
\begin{lstlisting}
(defun is-palindrome (lst)
    (equal lst (reverse lst)))
\end{lstlisting}
\end{minipage}

\section{Задание №2}

Написать предикат set-equal, который возвращает t, если два его
множества-аргумента содержат одни и те же элементы, порядок которых не имеет
значения.

\vspace{4mm}
\begin{minipage}{0.92\linewidth}
\begin{lstlisting}
(defun set-equal1 (set1 set2)
    (and (subsetp set1 set2) (subsetp set2 set1)))
\end{lstlisting}
\end{minipage}

\section{Задание №3}

Напишите свои необходимые функции, которые обрабатывают таблицу из 4-х точечных
пар (страна . столица) и возвращают по стране --- столицу, а по столице ---
страну.

\vspace{4mm}
\begin{minipage}{0.92\linewidth}
\begin{lstlisting}
(defun get-capital (table country)
    (cond ((null table) Nil)
          ((equal (caar table) country) (cdar table))
          (T (get-capital (cdr table) country))))
\end{lstlisting}
\end{minipage}

\vspace{4mm}
\begin{minipage}{0.92\linewidth}
\begin{lstlisting}
(defun get-country (table capital)
    (cond ((null table) Nil)
          ((equal (cdar table) capital) (caar table))
          (T (get-country (cdr table) capital))))
\end{lstlisting}
\end{minipage}

\section{Задание №4}

Напишите функцию swap-first-last, которая переставляет в списке-аргументе
первый и последний элементы.

\vspace{4mm}
\begin{minipage}{0.92\linewidth}
\begin{lstlisting}
(defun swap-first-last (lst)
    (cond ((null (cdr lst)) lst)
          (T (cons (car (last lst))
                  (reverse (cons (car lst) (cdr (reverse (cdr lst)))))))))
\end{lstlisting}
\end{minipage}

\section{Задание №5}

Напишите функцию swap-two-element, которая переставляет в списке-аргументе
два указанных своими порядковыми номерами элемента в этой списке.

\vspace{4mm}
\begin{minipage}{0.92\linewidth}
\begin{lstlisting}
(defun ste (lst num1 num2 cur-ind first-el res)
    (cond ((null lst) (reverse res))
          ((= cur-ind num1)
              (ste (cdr lst) num1 num2 (+ cur-ind 1) (car lst)
                   (cons (nth (- num2 num1) lst) res)))
          ((= cur-ind num2)
              (ste (cdr lst) num1 num2 (+ cur-ind 1) first-el
                   (cons first-el res)))
          (T (ste (cdr lst) num1 num2 (+ cur-ind 1) first-el
                  (cons (car lst) res)))))

(defun swap-two-element (lst num1 num2)
    (cond
     ((or (>= num1 (length lst))
          (>= num2 (length lst))
          (< num1 0)
          (< num2 0))
      Nil)
     ((< num1 num2) (ste lst num1 num2 0 Nil ()))
     (T (ste lst num2 num1 0 Nil ()))))
\end{lstlisting}
\end{minipage}

\section{Задание №6}

Напишите две функции swap-to-left и swap-to-right, которые производят одну
круговую перестановку в списке-аргументе влево и вправо, соответственно.

\vspace{4mm}
\begin{minipage}{0.92\linewidth}
\begin{lstlisting}
(defun swap-to-left (lst)
     (append (cdr lst) (cons (car lst) Nil)))
\end{lstlisting}
\end{minipage}

\vspace{4mm}
\begin{minipage}{0.92\linewidth}
\begin{lstlisting}
(defun swap-to-right (lst)
     (cons (car (last lst)) (reverse (cdr (reverse lst)))))
\end{lstlisting}
\end{minipage}

\section{Задание №7}

Напишите функцию, которая добавляет к множеству двухэлементных списков
новый двухэлементный список, если его там нет.

\vspace{4mm}
\begin{minipage}{0.92\linewidth}
\begin{lstlisting}
(defun adel (list-set lst is-in res)
    (cond ((null list-set) (cons (list lst) res))
          ; тут нужна проверка is-in еще вместе с null
          (T (add-two-el-list (cdr list-set) lst (or is-in (equal (car list-set)
          ) lst (cons (car list-set) res))))

(defun add-two-el-list (list-set lst)
    (adel list-set lst Nil ()))
\end{lstlisting}
\end{minipage}

\section{Задание №8}

Напишите функцию, которая умножает на заданное число-аргумент первый
числовой элемент списка из заданного 3-хэлементного списка-аргумента
когда:

\begin{enumerate}
    \item все элементы списка --- числа,

\vspace{4mm}
\begin{minipage}{0.92\linewidth}
\begin{lstlisting}
(defun mul-first-nums (lst num)
    (and lst (setf (car lst) (* (car lst) num)) lst))
\end{lstlisting}
\end{minipage}

    \item элементы списка --- любые объекты.

\vspace{4mm}
\begin{minipage}{0.92\linewidth}
\begin{lstlisting}
(defun mul-first-num (lst num)
    (cond ((null lst) lst)
          ((numberp (car lst)) (cons (* (car lst) num) (cdr lst)))
          (T (cons (car lst) (mul-first-num (cdr lst) num)))))
\end{lstlisting}
\end{minipage}

\end{enumerate}

\section{Задание №9}

Напишите функцию select-between, которая из списка-аргумента из 5 чисел
выбирает только те, которые расположены между двумя указанными
границами-аргументами и возвращет их в виде списка (упорядоченного по
возрастанию списка чисел).

\vspace{4mm}
\begin{minipage}{0.92\linewidth}
\begin{lstlisting}
(defun select-between (lst begin end)
    (cond ((null lst) Nil)
          ((not (< begin (car lst) end))
                (select-between (cdr lst) begin end))
          (T (cons (car lst) (select-between (cdr lst) begin end)))))

(defun sort-select-between (lst begin end)
    (sort (select-between lst begin end) #'<))
\end{lstlisting}
\end{minipage}

\chapter{Теоретические вопросы}

\section{Структуроразрушающие и не разрушающие структуру списка
функции}

Структуроразрушающие функции --- функции, после использования которых
теряется возможность работы изначальными списками. Эти функции изменяют сам
объект. Чаще всего такие функции начинаются с префикса \texttt{n} (например,
\texttt{nconc}, \texttt{nreverse} и т.~п.).

Функции, не разрушающие структуру списка --- функции, после использования
которых сохраняется возможность работы с изначальными списками. Эти функции
не изменяют сам объект, а создают копии (например, \texttt{append},
\texttt{reverse}, \texttt{length} и т.~п.).

\section{Отличие в работе функций cons, list, append, nconc и в их результате}

\texttt{conc} создает списковую ячейку и расставляет указатели (car-указатель
--- на первый аргумент, cdr-указатель --- на второй аргумент), если второй
аргумент --- список, то возвращает список, если нет --- точечную пару.

\texttt{list} создает столько списковых ячеек, сколько аргументов, возвращет
список.

\texttt{append} создает копии всех элементов кроме последнего и расставляет
cdr-указатели последних элементов копий на следующие копии/аргумент.

\texttt{nconc} аналогична \texttt{append} только работает не с копиями, а с
исходными списками.
