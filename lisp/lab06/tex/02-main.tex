\chapter{Практические задания}

\section{Задание №1}

Напишите функцию, которая уменьшает на 10 все числа из списка-аргумента этой
функции.

\vspace{4mm}
\hfill
\begin{minipage}{0.92\linewidth}
\begin{lstlisting}
(defun minus10 (lst)
    (mapcar #'(lambda (x)
                 (cond ((numberp x) (- x 10))
                       ((atom x) x)
                       (T (minus10 x)))) lst))
\end{lstlisting}
\end{minipage}

\section{Задание №2}

Напишите функцию, которая умножает на заданное число-аргумент все числа
из заданного списка-аргумента, когда:

\begin{enumerate}
    \item все элементы списка --- числа,

\vspace{4mm}
\begin{minipage}{0.92\linewidth}
\begin{lstlisting}
(defun mul-num (lst num)
    (mapcar #'(lambda (x) (* x num)) lst))
\end{lstlisting}
\end{minipage}

    \item элементы списка --- любые объекты.

\vspace{4mm}
\begin{minipage}{0.92\linewidth}
\begin{lstlisting}
(defun mul-num (lst num)
    (mapcar #'(lambda (x)
                 (cond ((numberp x) (* x num))
                       ((atom x) x)
                       (T (mul-num x num)))) lst))
\end{lstlisting}
\end{minipage}
\end{enumerate}

\section{Задание №3}

Написать функцию, которая по своему списке-аргументу lst определяет, является
ли он палиндромом (то есть равны ли lst и (reverse lst)).

\vspace{4mm}
\begin{minipage}{0.92\linewidth}
\begin{lstlisting}
(defun my-reverse (lst)
    (reduce #'(lambda (x y) (cons y x)) (cons nil lst)))

(defun is-palindrome (lst)
    (equal lst (my-reverse lst)))
\end{lstlisting}
\end{minipage}

\section{Задание №4}

Написать предикат set-equal, который возвращает t, если два его
множества-аргумента содержат одни и те же элементы, порядок которых не имеет
значения.

\vspace{4mm}
\begin{minipage}{0.92\linewidth}
\begin{lstlisting}
(defun is-in (el st)
    (reduce #'(lambda (x y) (or x y))
            (mapcar #'(lambda (x) (equal x el)) st) :initial-value Nil))

(defun is-subset (st1 st2)
    (reduce #'(lambda (x y) (and x y))
            (mapcar #'(lambda (x) (is-in x st2)) st1) :initial-value T))

(defun set-equal (st1 st2)
    (and (is-subset st1 st2) (is-subset st2 st1)))
\end{lstlisting}
\end{minipage}

\section{Задание №5}

Написать функцию, которая получает как аргумент список чисел, а возвращает
список квадратов этих чисел в том же порядке.

\vspace{4mm}
\begin{minipage}{0.92\linewidth}
\begin{lstlisting}
(defun squares (lst)
    (mapcar #'(lambda (x) (* x x) lst)))
\end{lstlisting}
\end{minipage}

\section{Задание №6}

Напишите функцию select-between, которая из списка-аргумента, содержащего
только числа, выбирает только те, которые расположены между двумя указанными
границами-аргументами и возращает их в виде списка.

\vspace{4mm}
\begin{minipage}{0.92\linewidth}
\begin{lstlisting}
(defun select-between (lst begin end)
    (remove-if-not #'(lambda (x) (< begin x end)) lst))
\end{lstlisting}
\end{minipage}

\section{Задание №7}

Написать функцию, вычисляющую декартово произведение двух своих
списков-аргументов.

\vspace{4mm}
\begin{minipage}{0.92\linewidth}
\begin{lstlisting}
(defun cartesian (lst1 lst2)
    (mapcan #'(lambda (el1)
                    (mapcar #'(lambda (el2) (list el1 el2)) lst2)) lst1))
\end{lstlisting}
\end{minipage}

\section{Задание №8}

Почему так реализовано reduce, в чем причина?

\vspace{4mm}
\begin{minipage}{0.92\linewidth}
\begin{lstlisting}
(reduce #'+ '(0)) ; -> 0
(reduce #'+ 0)    ; -> Error
\end{lstlisting}
\end{minipage}

Функция проверяет список-аргумент. Если он содержит только один элемент и
initial-value не задано, то возвращается элемент и функция не вызывается.
Если список-аргумент пустой и initial-value не задано, то функция вызывается с
нулевым количеством параметров.

\section{Задание №9}

Пусть list-of-list список, состоящий из списков. Написать функцию, которая
вычисляет сумму длин всех элементов list-of-list, т.~е., например, для
аргумента ((1 2) (3 4)) -> 4.

\vspace{4mm}
\begin{minipage}{0.92\linewidth}
\begin{lstlisting}
(defun get-length (lst)
    (reduce #'+ (mapcar #'(lambda (x) (cond ((atom x) 1)
                                            (T (get-length x)))) lst)))
\end{lstlisting}
\end{minipage}

