\chapter{Практические задания}

\vspace{4mm}
\hfill
\begin{minipage}{0.92\linewidth}
\begin{lstlisting}
\end{lstlisting}
\end{minipage}

\section{Задание №1}

Напишите функцию, которая уменьшает на 10 все числа из списка-аргумента этой
фукнции.

\vspace{4mm}
\hfill
\begin{minipage}{0.92\linewidth}
\begin{lstlisting}
\end{lstlisting}
\end{minipage}

\section{Задание №2}

Напишите функцию, которая умножает на заданное число-аргумент все числа
из заданного списка-аргумента, когда:

\begin{enumerate}
    \item все элементы списка --- числа,

\vspace{4mm}
\begin{minipage}{0.92\linewidth}
\begin{lstlisting}
\end{lstlisting}
\end{minipage}

    \item элементы списка --- любые объекты.

\vspace{4mm}
\begin{minipage}{0.92\linewidth}
\begin{lstlisting}
\end{lstlisting}
\end{minipage}
\end{enumerate}

\section{Задание №3}

Написать функцию, которая по своему списке-аргументу lst определяет, является
ли он палиндромом (то есть равны ли lst и (reverse lst)).

\vspace{4mm}
\begin{minipage}{0.92\linewidth}
\begin{lstlisting}
\end{lstlisting}
\end{minipage}

\section{Задание №4}

Написать предикат set-equal, который возвращает t, если два его
множества-аргумента содержат одни и те же элементы, порядок которых не имеет
значения.

\vspace{4mm}
\begin{minipage}{0.92\linewidth}
\begin{lstlisting}
\end{lstlisting}
\end{minipage}

\section{Задание №5}

Написать фукнцию, которая получает как аргумент список чисел, а возвращает
список квадратов этих чисел в том же порядке.

\vspace{4mm}
\begin{minipage}{0.92\linewidth}
\begin{lstlisting}
\end{lstlisting}
\end{minipage}

\section{Задание №6}

Напишите функцию select-between, которая из списка-аргумента, содержащего
только числа, выбирает только те, которые расположены между двумя указанными
границами-аргументами и возращает их в виде списка (упорядоченного по
возрастанию списка чисел).

\vspace{4mm}
\begin{minipage}{0.92\linewidth}
\begin{lstlisting}
\end{lstlisting}
\end{minipage}

\section{Задание №7}

Написать функцию, вычисляющую декартово произведение двух своих
списков-аргументов.

\vspace{4mm}
\begin{minipage}{0.92\linewidth}
\begin{lstlisting}
\end{lstlisting}
\end{minipage}

\section{Задание №8}

Почему так реализовано reduce, в чем причина?

\vspace{4mm}
\begin{minipage}{0.92\linewidth}
\begin{lstlisting}
(reduce #'+0) -> 0
(reduce #'+()) -> 0
\end{lstlisting}
\end{minipage}

\vspace{4mm}
\begin{minipage}{0.92\linewidth}
\begin{lstlisting}
\end{lstlisting}
\end{minipage}


\section{Задание №9}

Пусть list-of-list список, состоящий из списков. Написать фукнцию, которая
вычисляет сумму длин всех элементов list-of-list, т.~е., например, для
аргумента ((1 2) (3 4)) -> 4.

\vspace{4mm}
\begin{minipage}{0.92\linewidth}
\begin{lstlisting}
\end{lstlisting}
\end{minipage}

