\chapter{Практические задания}

\vspace{4mm}
\hfill
\begin{minipage}{0.92\linewidth}
\begin{lstlisting}
\end{lstlisting}
\end{minipage}

\section{Задание №1}

Написать хвостовую рекурсивную функцию my-reverse, которая развернет верхний
уровень своего списка-аргумента lst.

\vspace{4mm}
\hfill
\begin{minipage}{0.92\linewidth}
\begin{lstlisting}
(defun move-to (lst res)
    (cond ((null lst) res)
          (T (move-to (cdr lst) (cons (car lst) res)))))

(defun my-reverse (lst)
    (move-to lst ()))
\end{lstlisting}
\end{minipage}

\section{Задание №2}

Написать функцию, которая возвращает первый элемент списка-аргумента, который
сам является непустым списком.

\vspace{4mm}
\begin{minipage}{0.92\linewidth}
\begin{lstlisting}
(defun get-list (lst)
    (cond ((null lst) lst)
          ((and (car lst) (listp (car lst))) (car lst))
          (T (get-list (cdr lst)))))
\end{lstlisting}
\end{minipage}

\section{Задание №3}

Написать функцию, которая выбирает из заданного списка только те числа,
которые больше 1 и меньше 10 (между двумя заданными границами).

\vspace{4mm}
\begin{minipage}{0.92\linewidth}
\begin{lstlisting}
(defun sba (lst begin end res)
    (cond ((null lst) res)
          ((and (numberp (car lst)) (< begin (car lst) end))
               (sba (cdr lst) begin end (nconc res (list (car lst)))))
          ((listp (car lst)) 
               (sba (cdr lst) begin end (sba (car lst) begin end res)))
          (T (sba (cdr lst) begin end res))))

(defun select-between-all (lst begin end)
    (sba lst begin end ()))
\end{lstlisting}
\end{minipage}

\section{Задание №4}

Напишите рекурсивную функцию, которая умножает на заданное
число-аргумент все числа из заданного списка-аргумента, когда 

\begin{enumerate}
    \item все элементы списка --- числа,

\vspace{4mm}
\begin{minipage}{0.92\linewidth}
\begin{lstlisting}
(defun mul-nums (lst num)
    (cond (null lst) ))
\end{lstlisting}
\end{minipage}

    \item элементы списка --- любые объекты.

\vspace{4mm}
\begin{minipage}{0.92\linewidth}
\begin{lstlisting}
\end{lstlisting}
\end{minipage}

\end{enumerate}

\section{Задание №5}

Напишите функцию select-between, которая из списка-аргумента, содержащего только
числа выбирает только те, которые расположены между двумя указанными
границами-аргументами и возвращет их в виде списка (упорядоченного по
возрастанию списка чисел).

\vspace{4mm}
\begin{minipage}{0.92\linewidth}
\begin{lstlisting}
\end{lstlisting}
\end{minipage}

\section{Задание №6}

Написать рекурсивную версию (с именем rec-add) вычисления суммы чисел
заданного списка: 
\begin{enumerate}
    \item одноуровневого смешанного;

\vspace{4mm}
\begin{minipage}{0.92\linewidth}
\begin{lstlisting}
\end{lstlisting}
\end{minipage}

    \item структурированного.

\vspace{4mm}
\begin{minipage}{0.92\linewidth}
\begin{lstlisting}
\end{lstlisting}
\end{minipage}

\end{enumerate}

\section{Задание №7}

Написать рекурсивную версию с имененем recnth функции nth.

\vspace{4mm}
\begin{minipage}{0.92\linewidth}
\begin{lstlisting}
\end{lstlisting}
\end{minipage}

\section{Задание №8}

Написать рекурсивную функцию allodd, которая возвращает t, когда все элементы
списка нечетные.

\vspace{4mm}
\begin{minipage}{0.92\linewidth}
\begin{lstlisting}
\end{lstlisting}
\end{minipage}

\section{Задание №9}

Написать рекурсивную функцию, которая возвращает первое нечетное число
из списка (структурированного), возможно создавая некоторые вспомогательные
функции.

\vspace{4mm}
\begin{minipage}{0.92\linewidth}
\begin{lstlisting}
\end{lstlisting}
\end{minipage}

\section{Задание №10}

Используя cons-дополняемую рекурсию с одним тестом завершения, написать функцию,
которая получает как аргумент список чисел, а возвращает список квадратов этих
чисел в том же порядке.

\vspace{4mm}
\begin{minipage}{0.92\linewidth}
\begin{lstlisting}
\end{lstlisting}
\end{minipage}

